\documentclass[a4paper]{article}
\usepackage{multicol}
\usepackage[margin=1in]{geometry}
\usepackage{asymptote}
\usepackage{amsmath}
\usepackage{amssymb}
\usepackage{amsthm}
\newtheorem{theorem}{Theorem}
\newtheorem{lemma}[theorem]{Lemma}
\newtheorem{claim}[theorem]{Claim}
\everymath{\displaystyle}
\begin{document}
\begin{center}{\huge\textbf{Mock AusMC Intermediate\large\\30 problems}}\end{center}
\section{Problems}

\noindent\textbf{SHORT ANSWER: Give the numerical answer for each problem.}\begin{enumerate}
\item (Australian MC 2016 I / 5) What is the value of $(1\div2)\div(3\div4)$?
\item (Australian MC 2016 I / 6\footnote{Modified to fit for short answer.}) $0.75\%$ of a number is $6$. What is the number?
\item (Australian MC 2016 I / 3\footnote{Modified to fit for short answer.}) The cycling road race through the Adelaide Hills started at 11:15 am and the winner finished at 2:09 pm the same day. What is the winner's time in minutes?
\item (Australian MC 2016 I / 2\footnote{Modified wording and diagram.}) In the figure, the unlabeled region is what fraction of the circle? 
\begin{center}
\begin{asy}
import graph; usepackage("amsmath"); size(5cm); real lsf=0.5; pen dps=linewidth(0.7)+fontsize(10); defaultpen(dps); pen ds=black; real xmin=-7.22,xmax=7.94,ymin=-7.24,ymax=11.52; draw(circle((0.,4.),4.),linewidth(2.)); draw((0.,4.)--(1.2360679774997898,0.19577393481938588),linewidth(2.)); draw((0.,4.)--(0.,0.),linewidth(2.)); draw((0.,4.)--(-4.,4.),linewidth(2.)); draw((0.,4.)--(4.,4.),linewidth(2.)); draw(shift((0.,4.))*xscale(4.)*yscale(4.)*arc((0,0),1,270.,288.)--(0.,4.)--cycle,linewidth(2.)); label("$\dfrac12$",(-0.24,6.92),SE*lsf); label("$\dfrac14$",(-2.02,3.20),SE*lsf); label("$\dfrac15$",(1.78,3.20),SE*lsf); clip((xmin,ymin)--(xmin,ymax)--(xmax,ymax)--(xmax,ymin)--cycle);
\end{asy}
\end{center}
\item (Australian MC 2016 I / 1) What is the value of $20\times16$?
\item (Australian MC 2016 I / 4\footnote{Modified to fit for short answer.}) The fraction $\frac{720163}{2016}$ is between which two powers of 10? (Give the numbers themselves and not the exponents.)
\item (Australian MC 2016 I / 8\footnote{Modified without diagram.}) Consider the descriptions of four squares P, Q, R, and S as follows.
\begin{itemize}
\item In square P, the distance between the center and a vertex is 1 unit.
\item In square Q, the distance between the center and a side is 1 unit.
\item In square R, the distance between two opposite vertices is 1 unit.
\item In square S, the side length is 1 unit.
\end{itemize}
Which of the squares would have the greatest perimeter?
\item (PMO 2024 Qualifying / I.4) Let $a$ and $b$ be the last two digits of the 5-digit number $\overline{764ab}$. What is the largest possible value of the product of $ab^2$ if the 5-digit number is divisible by 6?
\item (AHSME 1978 / 11\footnote{Modified to fit for short answer.}) If $r$ is positive and the line whose equation is $x + y = r$ is tangent to the circle whose equation is $x^2 + y ^2 = r$, then what is the value of $r$?
\item (PMO 2024 Qualifying / I.1) Let $a,b,c$ be positive integers such that $$3.14=a+\frac1{b+\frac1c}.$$ What is $a+b+c?$
\item (Paraguay 2015 / 1) Alexa wrote the first 16 numbers of a sequence: \begin{equation*}1, 2, 2, 3, 4, 4, 5, 6, 6, 7, 8, 8, 9, 10, 10, 11, \dots\end{equation*} Then she continued following the same pattern, until she had 2015 numbers in total. What was the last number she wrote?
\item (Australian MC 2016 I / 7) In the expression below, the letters $A,B,C,D,$ and $E$ represent the numbers $1,2,3,4,$ and $5$ in some order. $$A\timesB+C\timesD+E$$ What is the largest value of the expression?
\item (Mathleague 12300 HS Relay / 1-1) Let $K$ be the number of prime numbers $p$ such that $p$ is a factor of $(T+1)!$, but $p^2$ is not. Find $K$.
\item (Sipnayan 2023 JHS Semifinals / E5) What is the least number of colors needed to color a $100\times100$ chessboard such that no two vertically, horizontally, and diagonally adjacent squares have the same color?
\item (Australian MC 2016 I / 10) There are 3 blue pens, 4 red pens and 5 yellow pens in a box. Without looking, I take pens from the box one by one. How many pens do I need to take from the box to be certain that I have at least one pen of each colour?
\item (PMO 2024 Qualifying / I.5) An urn contains two white and two black balls. John draws two balls simultaneously from the urn. If the balls are of different colors, he stops. Otherwise, he returns both balls to the urn and then repeats the process. What is the probability that he stops after exactly three draws?
\item (Mathleague 12300 HS Relay / 3-1) Let $K$ be the 65th digit after the decimal point in the decimal expansion of $\frac1{17}$. Find $K$.
\item (PMO 2024 Qualifying / I.2) What is the area of a rhombus whose diagonals have length 14 and 48, respectively?
\item (PMO 2024 Quailfying / I.3) The arithmetic mean of 11 integers is 10. After adding 20 to each of the first four and subtracting 24 from each of the last seven, what is the new mean?
\item (Australian MC 2016 I / 9\footnote{Modified without diagram and to fit for short answer.}) On a clock face, a line is drawn between 9 and 3 and another between 12 and 8. What is the acute angle between these lines in degrees?
\item (PMO 2024 Qualifying / II.5) Let $r$ and $s$ of the polynomial $x^2+2x+3$. What is the value of $\frac1{r^2-1}+\frac1{s^2-1}$?
\item (T. Andreescu, \textit{105 Algebra Problems}, Chapter 2) If $a$ is a real number such that $a-\frac1a=2,$ find $a^4+\frac1{a^4}.$
\item (Australian MC 2016 S / 29 (I / 30)) Around a circle, I place 64 equally spaced points, so that there are $64\times63\div2 = 2016$ possible chords between these points. I draw some of these chords, but each chord cannot cut across more than one other chord. What is the maximum number of chords I can draw?
\item (PMO 2024 Qualifying / II.3) Find the remainder when $\sum^{2023}_{n=1}2023^n$ is divided by 15.
\item (PMO 2024 Qualifying / II.6) Eight people are to sit around a round table on equally-spaced seats. Two of the eight people, Alice and Bob, insist on sitting next to each other. Meanwhile, another two, Clara and Dan, insist on sitting opposite each other. How many ways are there to seat the eight people? Rotations are considered equivalent.
\item (Sipnayan 2023 JHS Semifinals / A2) Let $X$ be the product of all odd integers from 1 to 49, inclusive. What are the last three digits of $X?$
\item (Sipnayan 2023 JHS Semifinals / E1) What is the area of a regular dodecagon (a 12-sided polygon) inscribed in a circle of diameter 12?
\item (PMO 2024 Qualifying / II.8\footnote{Modified diagram.}) A square of side length 1 and a rectangle of length 34 are inscribed in a semicircle as shown below. What is the area of the rectangle?
\begin{center}
\begin{asy}
import graph; size(7.6cm); real lsf=0.5; pen dps=linewidth(0.7)+fontsize(10); defaultpen(dps); pen ds=black; real xmin=-3.,xmax=35.,ymin=-7.,ymax=13.; draw((-2.,-6.)--(-1.,-6.)--(-1.,-5.)--(-2.,-5.)--cycle,linewidth(1)); draw((-1.,-6.)--(-1.,0.)--(33.,0.)--(33.,-6.)--cycle,linewidth(1)); draw((-2.,-6.)--(-1.,-6.),linewidth(1)); draw((-1.,-6.)--(-1.,-5.),linewidth(1)); draw((-1.,-5.)--(-2.,-5.),linewidth(1)); draw((-2.,-5.)--(-2.,-6.),linewidth(1)); draw((-1.,-6.)--(-1.,0.),linewidth(1)); draw((-1.,0.)--(33.,0.),linewidth(1)); draw((33.,0.)--(33.,-6.),linewidth(1)); draw((33.,-6.)--(-1.,-6.),linewidth(1)); draw(shift((16.,-6.))*xscale(18.027756377319946)*yscale(18.027756377319946)*arc((0,0),1,0.,180.),linewidth(1)); draw((-2.0277563773199496,-6.)--(34.027756377319946,-6.),linewidth(1)); clip((xmin,ymin)--(xmin,ymax)--(xmax,ymax)--(xmax,ymin)--cycle);
\end{asy}
\end{center}
\item (PMO 2024 Qualifying / II.1) Today --- the 2nd day of the 12th month of the year 2023 --- marks the start of the 26th Philippine Mathematical Olympiad. What is the remainder when $2023^{12^2}$ is divided by 26?
\item (Sipnayan 2023 JHS Semifinals / A4) Anna the Anaconda can clean her burrow in $x$ hours, while Sally the Spider can do the same in $y$ hours. If they can finish cleaning it together in $4$ hours, and $x$ and $y$ are positive integers, then what is the largest possible value of $x+y?$\end{enumerate}\begin{center}\textbf{--- THE ANSWERS ARE ON THE FOLLOWING PAGE ---}\end{center}\newpage
\section{Answers and Solutions}
\noindent\textbf{SHORT ANSWER:}\begin{multicols}{2}\begin{enumerate}
\item $\frac23$
\item 800
\item 174
\item $\frac1{20}$
\item 320
\item 100 and 1000
\item Q
\item 512
\item 2
\item 17
\item 1344
\item 27
\item 2
\item 4
\item 10
\item $\frac{2}{27}$
\item 0
\item 336
\item 2
\item $60^\circ$
\item $-\frac13$
\item 7
\item 156
\item 4
\item 192
\item 625
\item 108
\item 204
\item 1
\item 25\end{enumerate}\end{multicols}
\end{document}